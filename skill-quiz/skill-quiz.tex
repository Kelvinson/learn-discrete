\documentclass[12pt]{article}
\usepackage{fancyhdr}
\usepackage[margin=1.9cm]{geometry}
\usepackage{parskip}
\usepackage{times} 
\usepackage{tabularx}
\usepackage{enumitem}
\usepackage{nicefrac}
\usepackage{amssymb, amsmath}
\usepackage{graphicx,xcolor}
\definecolor{mydarkblue}{rgb}{0,0.08,0.45}
\usepackage{hyperref}
\usepackage{acronym}

\newcommand{\acro}[1]{\textsc{#1}}
\hypersetup{ %
    colorlinks=true,
    citecolor=mydarkblue,
    urlcolor=mydarkblue,
    filecolor=mydarkblue,
    linkcolor=mydarkblue}

\newcommand{\vx}{{\mathbf{x}}}
\newcommand{\vy}{{\mathbf{y}}}

\title{\vspace{-4ex}Background Knowledge Quiz}

\date{\vspace{-4ex}}

\begin{document}
\pagenumbering{gobble}

\maketitle

\begin{enumerate}
\item Your name:
\item What year and program are you in?
\item Are you taking this course for credit, auditing, or on the waiting list?
\subsubsection*{Gaussians}
\item If $p(x) = \mathcal{N}(x | \mu, \sigma^2)$,
\begin{enumerate}
\item For some $x \in \mathbb{R}, \mu \in \mathbb{R}, \sigma \in \mathbb{R}^+$, can $p(x) < 0$?
\item For some $x \in \mathbb{R}, \mu \in \mathbb{R}, \sigma \in \mathbb{R}^+$, can $p(x) > 1$?
\end{enumerate}
\item If $p(x) = \mathcal{N}(x | \mu, \Sigma)$ with $x \in \mathbb{R}^D, \mu \in \mathbb{R}^D, \Sigma \in \mathbb{R}^{D\times D}$, (a multivariate Gaussian),
\begin{enumerate}
\item What is the computational complexity (the asymptotic time cost) of evaluating $p(x)$?
\item What restrictions are there on $\Sigma$ in order for it to be a valid covariance matrix?
\end{enumerate}
\subsubsection*{Derivatives}
\item If $A$ is a fixed matrix and $\vx$ is a vector, what is $\frac{\partial (A \vx)_i}{\partial \vx_j}$?
\item Given a composition of functions $f(x) = a(b(c(x)))$, we can evaluate its derivative using the chain rule - just multiply together the Jacobian of each function.
What is the fastest order to multiply this product of Jacobians $J_a \times J_b \times J_c$, if $f(x)$ is a vector-input, scalar-output function?\vspace{0.5cm}
\item How could one form an unbiased estimate of $\nabla_x \int f(x, \theta) p(\theta) d\theta$ using samples from $p(\theta)$, and derivatives of $f$?
%\item Write a function from $\mathbb{R} \rightarrow (0,1)$ that is monotonic and differentiable everywhere.
\subsubsection*{Distributions}
\item In the natural exponential family of distributions, $p(x|\theta) = f(x)g(\theta)\exp\{x\theta\}$.
What must $g(\theta)$ be in order for $p(x | \theta)$ to be a valid probability distribution?\vspace{0.5cm}
\item One way to specify a Categorical (discrete) distribution using an unconstrained vector $\vx \in \mathbb{R}^D$ is with the softmax function: $p(y = c | \vx) = \frac{\exp\{x_c\}}{\sum_{c'=1}^D \exp\{x_{c'}\}}$:
\begin{enumerate}
\item What could go wrong numerically in evaluating $p(y = c | \vx)$ if some elements of $\vx$ are large?\vspace{0.5cm}
\item How could one fix this?
\end{enumerate}
\end{enumerate}
\end{document}